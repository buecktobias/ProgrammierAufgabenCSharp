\documentclass[12pt]{article}
\usepackage{amsmath, amssymb, xcolor, fancyhdr, enumitem, geometry, listings, titling, setspace, microtype, titlesec, hyperref, parskip}
\usepackage[utf8]{inputenc}
\usepackage[T1]{fontenc}
\usepackage[ngerman]{babel}
\usepackage{changepage} % For indentation

% Page geometry
\geometry{a4paper, top=2cm, left=2cm, right=2cm, bottom=2cm}

% Font selection: Modern sans-serif font
\usepackage{helvet}
\renewcommand{\familydefault}{\sfdefault} % Use sans-serif globally

% Header and footer styling
\pagestyle{fancy}
\fancyhf{}
\fancyhead[L]{\textbf{Programmieren Tutorium}}
\fancyhead[R]{\textbf{Tobias Bück, Sam Haghighi}}
\fancyfoot[C]{\thepage}
\fancyfoot[C]{%
  \raisebox{2em}{\parbox{\textwidth}{%
      \small © Tobias Bück, CC BY 4.0.%
  }}%
  \thepage
}
\renewcommand{\headrulewidth}{0.4pt}
\renewcommand{\footrulewidth}{0.4pt}

% Title customization
\pretitle{\vspace{-2em} \begin{center}\LARGE\bfseries}
\posttitle{\end{center}\vspace{1em}}
\preauthor{}
\postauthor{}
\predate{}
\postdate{}

% Author and date styling
\renewcommand{\maketitle}{
    \begin{center}
        {\LARGE \bfseries \thetitle \par}
        \vspace{0.25em}
        {\small Datum: \today \par}
        \vspace{0.5em}
    \end{center}
}



% Custom colors and refined code style for C#
\definecolor{lightgray}{gray}{0.95}
\definecolor{darkblue}{rgb}{0, 0.2, 0.5}
\definecolor{darkgreen}{rgb}{0, 0.5, 0.2}
\definecolor{keywordcolor}{rgb}{0, 0.4, 0.8}
\definecolor{stringcolor}{rgb}{0.5, 0, 0.5}
\definecolor{commentcolor}{rgb}{0.25, 0.5, 0.25}
\definecolor{identifiercolor}{rgb}{0.6, 0.2, 0.2}

\lstset{
    language=[Sharp]C,
    basicstyle=\small\ttfamily,
    tabsize=3,
    breaklines=true,
    breakatwhitespace=true,
    frame=tb,
    framesep=6pt,
    rulecolor=\color{gray},
    backgroundcolor=\color{lightgray},
    stringstyle=\color{stringcolor}\ttfamily,
    commentstyle=\color{commentcolor}\ttfamily,
    keywordstyle=\color{keywordcolor}\bfseries,
    identifierstyle=\color{identifiercolor},
    showstringspaces=false,
    showspaces=false,
    showtabs=false,
    xleftmargin=12pt,
    framexleftmargin=12pt,
    morecomment=[l]{//},
    morecomment=[s]{/*}{*/},
    morekeywords={abstract, event, new, struct, var,
        as, explicit, null, switch,
        base, extern, object, this,
        bool, false, operator, throw,
        break, finally, out, true,
        byte, fixed, override, try,
        case, float, params, typeof,
        catch, for, private, uint,
        char, foreach, protected, ulong,
        checked, goto, public, unchecked,
        class, if, readonly, unsafe,
        const, implicit, ref, ushort,
        continue, in, return, using,
        decimal, int, sbyte, virtual,
        default, interface, sealed, volatile,
        delegate, internal, short, void,
        do, is, sizeof, while,
        double, lock, stackalloc,
        else, long, static,
        enum, namespace, string},
}

% Custom environments and commands
\newenvironment{aufgabe}[2]
  {%
   \section*{Aufgabe #1: #2}
   \vspace{0.5em}
   \begin{adjustwidth}{1cm}{}  % Indent by 1 cm
  }
  {%
   \end{adjustwidth}
  }
  
\newcommand{\subaufgabe}[1]{
    #1
}

\newcommand{\hint}[1]{
    \vspace{0.5em}
    \textbf{\textit{Hinweis:}} #1
    \vspace{0.5em}
}

\begin{document}

\title{Programmieren Tutorium \\ \large Arbeitsblatt 6}
\author{Tobias Bück, Sam Haghighi}
\date{\today}
\maketitle

\begin{aufgabe}{1}{Klausuraufgaben aus Programmieren I WS 2018}
\textbf{Übersetzen Sie die folgenden kursiv gedruckten Sätze in C\#-Anweisungen.}

\subaufgabe{a) Die Variablen \texttt{i1} und \texttt{i2} enthalten ganze Zahlen.}
\textit{Wenn die Differenz aus \texttt{i1} und \texttt{i2} kleiner oder gleich 2 ist, dann wird die Hälfte von \texttt{i1} ausgegeben. Andernfalls wird \texttt{i2} verdoppelt.}

\subaufgabe{b) Die Variable \texttt{d1} ist eine Gleitpunktzahl.}
\textit{Solange \texttt{d1} größer als 5 ist, wird das gemacht: Der Wert von \texttt{d1} wird um 3 vermindert. Wenn dann das Quadrat von \texttt{d1} kleiner als 10 ist, wird das Dreifache von \texttt{d1} ausgegeben.}

\subaufgabe{c) Die Variable t1 ist vom Typ string, anz von Typ int. Übersetzen Sie: Für jedes
Element des Strings wird das gemacht: Wenn der ASCII-Wert des Elements gerade ist,
wird der Wert von anz um 3 erhöht.}

\subaufgabe{d) Die Variable iifeld ist vom Typ List<int>. Übersetzen Sie: Für alle Elemente von
iifeld vom ersten bis zum letzten wird das gemacht: Wenn das aktuelle Element eine
Zahl zwischen 65 und 91 (beides einschlieÿlich) ist, wird der Buchstabe ausgegeben, dessen
ASCII-Wert dem Zahlenwert entspricht.}
\\
\\
\hint{
Mit (char) <INTEGER> bekommst du den ASCII Buchstaben von einer Zahl. 
z.B (char) 65 => A.
}
\\
\subaufgabe{e) Die Variable \texttt{bfeld} ist vom Typ \texttt{bool[]}.}
\textit{Für jedes fünfte Element von \texttt{bfeld} wird das gemacht: Wenn der Index durch 2 teilbar ist, wird das Element auf \texttt{true} gesetzt, sonst auf \texttt{false}.}
\end{aufgabe}

\begin{aufgabe}{2}{structs/Klassen definieren}
\hint{In C\# wird ein \texttt{struct} wie folgt definiert:}
\begin{lstlisting}
public struct NameDesStructs {
    public TYP AttributName1;
    public TYP AttributName2;
}
\end{lstlisting}
\newpage
\hint{In C\# wird eine Instanz/Objekt einer Klasse/struct wie folgt erstellt:}
\begin{lstlisting}
var variablenName = new NameDesStructs();
\end{lstlisting}
\texttt{Oder so:}
\begin{lstlisting}
NameDesStructs variablenName = new NameDesStructs();
\end{lstlisting}
\subaufgabe{a)
Definiere eine Struktur/struct/Klasse \texttt{Person}. Erstelle dann zwei Personen-Objekte/Instanzen.}

\subaufgabe{b) Füge ein Feld \texttt{Name} zur Struktur/struct/Klasse \texttt{Person} hinzu. Erstelle dann zwei Personen Objekte und weise ihnen verschiedene Namen zu. Gib die Namen der beiden Personen auf der Konsole aus.}

\subaufgabe{c) Struct \texttt{Person} mit Name und Alter: \\
Erweitere die Struktur \texttt{Person} um ein weiteres Feld \texttt{Alter}. Erstelle ein \texttt{Person}-Objekt mit dem Namen \"Hans\" und einem Alter von 20. Simuliere einen Geburtstag, indem du das Alter um 1 erhöhst, und gib das Alter vor und nach dem Geburtstag auf der Konsole aus.

}

\subaufgabe{d) Vergleich von zwei Personen nach Alter. \\
Verwende die Struktur \texttt{Person} mit den Feldern \texttt{Name} und \texttt{Alter}. Frage den Namen und das Alter von zwei Personen ab. Vergleiche das Alter der beiden Personen und gib aus: Person X ist älter als Person Y
 oder Person X und Y sind gleich alt. (Entweder direkt in der Main oder schreibe eine einfache statische Methode}


\subaufgabe{e) Liste von Personen (anspruchsvoll)\\}
\hint{ In C\# wird eine Liste von Personen wie folgt erstellt}
\begin{lstlisting}
var personenListe = new List<Person>();
\end{lstlisting}
Erstelle eine leere Liste von Personen. Füge anschließend zwei Personen zur Liste hinzu: Martin mit dem Alter 22 und Julian mit dem Alter 40. Gib das Alter der Person an Position 0 in der Liste, auf der Konsole aus.

\subaufgabe{f) Suche in der Liste (schwer)\\
Du hast eine Liste von Personen. Schreibe eine statische Funktion, die überprüft, ob eine Person mit einem bestimmten Alter X in der Liste existiert.}

\subaufgabe{g) Verbessertes Personen erstellen (schwer): \\
Schreibe eine Konstruktor für Person. An diesen soll der Name und das Alter übergeben werden.
}

\hint{In C\# definiert man einen Konstruktor folgendermaßen:}
\begin{lstlisting}
public struct Person{
    // Attribute
    public Person(TYP parameterName1, TYP parameterName2){
        // Initialisieren des Objekts, Erstellen des Objekts
    }
}
\end{lstlisting}

\subaufgabe{h) Personen ausgeben (schwer): \\
Implementiere eine ToString Methode in der Klasse \texttt{Person}. Diese soll ausgeben: Hans ist 18 Jahre alt.
}

\hint{Um die Ausgabe eines Objekts zu verändern, definiert man die ToString Methode}
\begin{lstlisting}
public struct Person{
    public override string ToString(){
        // Wichtig du musst einen String returnen, keine Console.WriteLine Statements
        // Meistens gibt man die Attribute an! Z.B
        return $"Person: Vorname {this.FirstName}"
    }
}
class Program{
    public static void Main(){
        var person = new Person("Hans", 20);
        Console.WriteLine(person);
    }
}
\end{lstlisting}

\end{aufgabe}

\begin{aufgabe}{3}{structs und Funktionen}

\subaufgabe{a) Struct \texttt{Handy} mit Ladestand}
Erstelle eine Struktur \texttt{Handy} mit einem Feld \texttt{Akkustand}, der den aktuellen Ladestand in Prozent von 0 bis 100 speichert. Erstelle ein \texttt{Handy}-Objekt und setze den Akkustand auf 100.

\subaufgabe{b) Methode zum Aufladen des Akkus \\
Schreibe eine Methode \texttt{AkkuAufladen}, die den Akkustand des Handys um 1\% erhöht.
}

\subaufgabe{c) Verhindere, dass der Akkustand über 100\% steigt, indem die Funktion \texttt{AkkuAufladen} aus \textit{b)} bei 100\% keine weitere Erhöhung vornimmt.
}

\subaufgabe{d) Liste von Handys
Erstelle eine Liste von Handys und füge 3 \texttt{Handy}-Objekte mit verschiedenen Akkuständen hinzu.
Lade das Handy an Index 2 der Liste auf. Gebe den Akku Stand des Handys vor und nach dem Laden aus.
}

\hint{Erstellen einer Handy Liste:}
\begin{lstlisting}
var handys = new List<Handy>();
var handy1 = new Handy();
handy1.Akku = 20; 
\end{lstlisting}

\subaufgabe{e) Erstellen von 100 Handys mit unterschiedlichen Akkuständen (anspruchsvoll) \\
Erstelle eine Liste mit 100 \texttt{Handy}-Objekten. Das erste Handy soll einen Akkustand von 0 haben, das zweite 1, das dritte 2 usw., bis zum letzten Handy mit einem Akkustand von 99.
}

\subaufgabe{f) Schreibe einen Konstruktor und ToString (schwer)\\
Überlege dir einen sinnvollen Konstruktor und eine ToString MEthode für das Handy.
}

\subaufgabe{d) Größter Akkustand in der Liste (schwer):\\
Finde den größten Akkustand in der Liste von Handys und gib den Akkustand des Handys mit der höchsten Ladung aus.
}

\end{aufgabe}

\begin{aufgabe}{4}{Erstelle eine Anwesenheitsliste}

Erstelle ein Anwesenheitsliste struct. Das struct soll eine Liste an Personen als Attribut haben.

\subaufgabe{a) Schreibe einen Konstruktor, welcher keine Argumente/Parameter übergeben bekommt}

\subaufgabe{b) Schreibe eine personHinzufügen(Person person) Methode/Funktion}

\subaufgabe{c) Schreibe eine istPersonAnwesend(Person person) Methode/Funktion}

\subaufgabe{d) Schreibe eine berechneAnzahlAnwesendePersonen() Methode/Funktion}

\end{aufgabe}

\end{document}
