\documentclass[12pt]{article}
\usepackage{amsmath, amssymb, xcolor, fancyhdr, enumitem, geometry, listings, titling, setspace, microtype, titlesec, hyperref, parskip}
\usepackage[utf8]{inputenc}
\usepackage[T1]{fontenc}
\usepackage[ngerman]{babel}
\usepackage{changepage} % For indentation
\usepackage{xcolor}
% Page geometry
\geometry{a4paper, top=2cm, left=1.5cm, right=1.5cm, bottom=2cm}

% Font selection: Modern sans-serif font
\usepackage{helvet}
\renewcommand{\familydefault}{\sfdefault} % Use sans-serif globally

% Header and footer styling
\pagestyle{fancy}
\fancyhf{}
\fancyhead[L]{\textbf{Programmieren Tutorium}}
\fancyhead[R]{\textbf{Tobias Bück, Sam Haghighi}}
\fancyfoot[C]{\thepage}
\fancyfoot[C]{%
  \raisebox{2em}{\parbox{\textwidth}{%
      \small © Tobias Bück, CC BY 4.0.%
  }}%
  \thepage
}
\renewcommand{\headrulewidth}{0.4pt}
\renewcommand{\footrulewidth}{0.4pt}

% Title customization
\pretitle{\vspace{-2em} \begin{center}\LARGE\bfseries}
\posttitle{\end{center}\vspace{1em}}
\preauthor{}
\postauthor{}
\predate{}
\postdate{}

% Author and date styling
\renewcommand{\maketitle}{
    \begin{center}
        {\LARGE \bfseries \thetitle \par}
        \vspace{0.25em}
        {\small Datum: \today \par}
        \vspace{0.5em}
    \end{center}
}



% Custom colors and refined code style for C#
\definecolor{lightgray}{gray}{0.95}
\definecolor{darkblue}{rgb}{0, 0.2, 0.5}
\definecolor{darkgreen}{rgb}{0, 0.5, 0.2}
\definecolor{keywordcolor}{rgb}{0, 0.4, 0.8}
\definecolor{stringcolor}{rgb}{0.5, 0, 0.5}
\definecolor{commentcolor}{rgb}{0.25, 0.5, 0.25}
\definecolor{identifiercolor}{rgb}{0.6, 0.2, 0.2}

\lstset{
    language=[Sharp]C,
    basicstyle=\small\ttfamily,
    tabsize=3,
    breaklines=true,
    breakatwhitespace=true,
    frame=tb,
    framesep=6pt,
    rulecolor=\color{gray},
    backgroundcolor=\color{lightgray},
    stringstyle=\color{stringcolor}\ttfamily,
    commentstyle=\color{commentcolor}\ttfamily,
    keywordstyle=\color{keywordcolor}\bfseries,
    identifierstyle=\color{identifiercolor},
    showstringspaces=false,
    showspaces=false,
    showtabs=false,
    xleftmargin=10pt,
    framexleftmargin=10pt,
    morecomment=[l]{//},
    morecomment=[s]{/*}{*/},
    morekeywords={abstract, event, new, struct, var,
        as, explicit, null, switch, Console,
        base, extern, object, this,
        bool, false, operator, throw,
        break, finally, out, true,
        byte, fixed, override, try,
        case, float, params, typeof,
        catch, for, private, uint,
        char, foreach, protected, ulong,
        checked, goto, public, unchecked,
        class, if, readonly, unsafe,
        const, implicit, ref, ushort,
        continue, in, return, using,
        decimal, int, sbyte, virtual,
        default, interface, sealed, volatile,
        delegate, internal, short, void,
        do, is, sizeof, while,
        double, lock, stackalloc,
        else, long, static,
        enum, namespace, string},
}

% Custom environments and commands
\newenvironment{aufgabe}[2]
  {%
   \section*{Aufgabe #1: #2}
   \vspace{0.5em}
   \begin{adjustwidth}{1cm}{}  % Indent by 1 cm
  }
  {%
   \end{adjustwidth}
  }
  
\newcommand{\subaufgabe}[1]{
    #1
}

\newcommand{\hint}[1]{
    \vspace{0.5em}
    \textbf{\textit{Hinweis:}} #1
    \vspace{0.5em}
}

\begin{document}

\title{Programmieren Tutorium \\ \large Arbeitsblatt 7}
\author{Tobias Bück, Sam Haghighi}
\date{\today}
\maketitle

\begin{aufgabe}{1}{Wie Klausuraufgabe 1}
Übersetzen Sie die folgenden Sätze in C\#-Anweisungen.

\subaufgabe{a)
Die Variablen z1 und z2 enthalten ganze Zahlen. Übersetzen Sie: Wenn z2 kleiner oder gleich
z1 und außerdem z2 größer als 4 ist, wird z1 der Kehrwert von z2 zugewiesen (also 1 geteilt durch z2). Dabei werden die Nachkommastellen abgeschnitten.}

\subaufgabe{b)
Die Variable st ist ein String. Übersetzen Sie: So lange die Länge von st nicht durch 7
teilbar ist, wird ein Fragezeichen an den String angehängt
}

\subaufgabe{c)
Die Variable df ist ein dynamisches Feld von Gleitpunktzahlen. Übersetzen Sie: Für alle
Zahlen des Feldes wird das Vierfache des Feldelementes ausgegeben.
}

\subaufgabe{d)
Die Variable sfeld ist ein Stringfeld vom Typ List<string>. Übersetzen Sie: Für jedes
Element des Feldes wird das gemacht: Wenn die Länge des Elements größer als 5 ist wird an das Element ein Punkt angehängt.
}
\end{aufgabe}
\begin{aufgabe}{2}{User - Konstruktor - To String - Properties}
\hint{In C\# wird ein \texttt{struct} wie folgt definiert:}
\begin{lstlisting}
public struct NameDesStructs {
    public TYP AttributName1;
    public TYP AttributName2;
}
\end{lstlisting}
\hint{In C\# wird eine Instanz/Objekt einer Klasse/struct wie folgt erstellt:}
\begin{lstlisting}
var variablenName = new NameDesStructs();
\end{lstlisting}
\texttt{Oder so:}
\begin{lstlisting}
NameDesStructs variablenName = new NameDesStructs();
\end{lstlisting}
\subaufgabe{a)
Definiere eine Struktur/struct/Klasse \texttt{User}. Erstelle dann zwei User-Objekte/Instanzen.}

\subaufgabe{b) Füge ein Feld \texttt{Vorname} zum struct \texttt{User} hinzu. Erstelle dann zwei User Objekte und weise ihnen verschiedene Vornamen zu. Gib die Vornamen der beiden Personen auf der Konsole aus.}

\subaufgabe{c) Füge ein Feld \texttt{Nachname} zum struct \texttt{User} hinzu.}


\subaufgabe{d) Verbessertes User erstellen: \\
Schreibe eine Konstruktor für User. An diesen soll der Vor und Nachname übergeben werden.
}

\hint{In C\# definiert man einen Konstruktor folgendermaßen:}
\begin{lstlisting}
public struct User{
    // Attribute
    public User(TYP parameterName1, TYP parameterName2){
        // Initialisieren des Objekts, Erstellen des Objekts
    }
}
\end{lstlisting}

\subaufgabe{e) Personen ausgeben: \\
Implementiere eine ToString Methode in der Klasse \texttt{Person}. Diese soll ausgeben: User: Udo Müller.
}

\hint{Um die Ausgabe eines Objekts zu verändern, definiert man die ToString Methode}
\begin{lstlisting}
public struct Person{
    public override string ToString(){
        // Wichtig du musst einen String returnen, keine Console.WriteLine Statements
        // Meistens gibt man die Attribute an! Z.B
        return $"Person: Vorname {this.FirstName}"
    }
}
class Program{
    public static void Main(){
        var person = new Person("Hans", 20);
        Console.WriteLine(person);
    }
}
\end{lstlisting}
\subaufgabe{f) GanzerName Property: \\
Implementiere eine Property Ganzer Name, welche den Ganzen Namen einer Property Zurückgibt. Über die Property Ganzer Name darf kein Wert gesetzt werden.
}
\\
\hint{
Nutze eine Property/Eigenschaft. Sie verhalten sich wie Variablen, sind jedoch sicherer und flexibler, da sie mit Logik ausgestattet werden können, um Werte zu prüfen oder zusätzliche Aktionen auszuführen.
}
\\
\textbf{Allgemein:}
\begin{lstlisting}
// In einem Struct oder einer Klasse
public TYP Prop
{
  get {  // get ist optional
    // Mache irgendetwas
    return someValue; 
  }
  set { // set ist optional
    // Bekommt unsichtbaren Parameter value
    // Mache irgendetwas Meistens veraendern von Variablen
  }
}
// In Main Methode
Console.WriteLine(meinStruct.Prop) // ruft get von Prop auf
meinStruct.Prop = 42; // ruft set von Prop mit value=42 auf

\end{lstlisting}

\newpage
\textbf{Beispiel: Wir haben eine Thermometer Klasse}
\begin{lstlisting}
struct Thermometer{
    public double celsius; 
    public double Fahrenheit
    {
      get { return (celsius * 9/5)+32; } //  Celsius -> Fahrenheit
      set { celsius = (value - 32)*5/9; } // Fahrenheit -> Celsius
    }
}
// In Main Methode
var thermometer = new Thermometer();
thermometer.celsius = 30;
Console.WriteLine(thermometer.Fahrenheit); // 86
thermometer.Fahrenheit = 77
Console.WriteLine(thermometer.celsius); // 25
\end{lstlisting}

\subaufgabe{g) GanzerName Property mit set: \\
Füge zu der GanzenNamen Property eine set Methode bzw. einen set Zugriff hinzu.
}
\\

\hint{Die Methode .Split(seperator) teilt einen Text in eine string Liste auf. Z.B:}
\begin{lstlisting}
var datum = "26.11.2024";

string[] datumsTeile = datum.Split(".");
string tag = datumsTeile[0]; // 26
string monat = datumsTeile[1]; // 11
string jahr = datumsTeile[2]; // 2024
\end{lstlisting}

\end{aufgabe}
\newpage
\begin{aufgabe}{3}{Dateien bearbeiten: Lesen, Zählen, Schreiben}
\hint{In C\# kann eine Datei mit der \texttt{StreamReader}- oder \texttt{StreamWriter}-Klasse verarbeitet werden. Nach dem Gebrauch ist es wichtig, die Datei zu schließen, um Ressourcen freizugeben.}
\\
\textbf{Dateien lesen}
\\
Zeichen für Zeichen:
\begin{lstlisting}
    var filePath = "sample.txt";
    var reader = new StreamReader(filePath);
    int character;
    while ((character = reader.Read()) != -1)
    {
        // Mache etwas!
    }
    reader.Close();
\end{lstlisting}

Zeile für Zeile:
\begin{lstlisting}
    var filePath = "sample.txt";
    var reader = new StreamReader(filePath);
    string line;
    while ((line = reader.ReadLine()) != null)
    {
        // Mache etwas!
    }
    reader.Close();
\end{lstlisting}
Nutze für folgende Aufgaben diese Beispiel Log Datei
\begin{lstlisting}[
    basicstyle=\fontsize{9pt}{11pt}\normalfont\ttfamily\color{darkgray},
    breaklines=true,
    frame=tb,
    framesep=2pt,
]
# Activity Log
Logged In
Error While User Logged In: Invalid Password
Logged In
Logged In
\end{lstlisting}
\subaufgabe{a) Dateiinhalt auf der Konsole ausgeben
Erstelle eine Methode, die den gesamten Inhalt einer Datei Zeile für Zeile auf der Konsole ausgibt.
}

\subaufgabe{b) Anzahl von Buchstaben L in einer Datei zählen \\
Erstelle eine Methode, die alle Buchstaben \texttt{L} in einer Datei zählt.
}

\subaufgabe{c) Anzahl von \texttt{Logged In} - Zeilen in einer Datei zählen \\
Erstelle eine Methode, die die Anzahl der Zeilen, welche exakt \texttt{Logged In} sind.
}

\subaufgabe{d) Eine Datei mit Zahlen befüllen \\
Erstelle eine Methode, die eine Datei \texttt{numbers.txt} erstellt und die Zahlen von 0 bis 1000 in die Datei schreibt.
}

\hint{Eine Datei erstellen, in eine Datei schreiben:}
\begin{lstlisting}
var filePath = "datei.txt";
var writer = new StreamWriter(filePath);
// Mache etwas: Write, WriteLine wie bei Console.Write
writer.Write("")
writer.WriteLine("")
writer.Close();
\end{lstlisting}

\end{aufgabe}

\begin{aufgabe}{4}{structs und Funktionen}

\subaufgabe{a) Struct \texttt{Handy} mit Ladestand}
Erstelle eine Struktur \texttt{Handy} mit einem Feld \texttt{Akkustand}, der den aktuellen Ladestand in Prozent von 0 bis 100 speichert. Erstelle ein \texttt{Handy}-Objekt und setze den Akkustand auf 100.

\subaufgabe{b) Methode zum Aufladen des Akkus \\
Schreibe eine Methode \texttt{AkkuAufladen}, die den Akkustand des Handys um 1\% erhöht.
}

\subaufgabe{c) Schreibe einen Konstruktor und ToString (schwer)\\
Überlege dir einen sinnvollen Konstruktor und eine ToString Methode für das Handy.
}

\end{aufgabe}

\newpage

\begin{aufgabe}{5}{Erstelle eine Anwesenheitsliste}

Erstelle ein Anwesenheitsliste struct. Das struct soll eine Liste an Personen als Attribut haben.

\hint{ Nutze folgendes Person struct}
\begin{lstlisting}
struct Person
{
    public string Name;
    
    public Person(string name)
    {
        Name = name;
    }

    public override string ToString()
    {
        return Name;
    }
}
\end{lstlisting}

\subaufgabe{a) Schreibe einen Konstruktor, welcher keine Argumente/Parameter übergeben bekommt}

\subaufgabe{b) Schreibe eine personHinzufügen(Person person) Methode/Funktion}

\subaufgabe{c) Schreibe eine sinnvolle ToString Methode}

\end{aufgabe}
\end{document}
