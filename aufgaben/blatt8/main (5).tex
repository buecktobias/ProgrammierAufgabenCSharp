\documentclass[12pt]{article}
\usepackage{amsmath, amssymb, xcolor, fancyhdr, enumitem, geometry, listings, titling, setspace, microtype, titlesec, hyperref, parskip}
\usepackage[utf8]{inputenc}
\usepackage[T1]{fontenc}
\usepackage[ngerman]{babel}
\usepackage{changepage} % For indentation
\usepackage{xcolor}
\usepackage{tcolorbox}
\usepackage{needspace}

\newtcolorbox{notebox}{
    colback=blue!10!white, 
    colframe=blue!50!black, 
    title=Wichtige Information, 
    fonttitle=\bfseries,
    boxrule=0.5mm,       % Thickness of the frame
    arc=3mm,             % Rounding of corners
    boxsep=6pt,          % Padding inside the box
    toptitle=2pt,        % Extra space above the title
    bottomtitle=1pt      % Extra space below the title
}

\newcommand{\note}[1]{
    \begin{notebox}
    #1
    \end{notebox}
}

% Page geometry
\geometry{a4paper, top=2cm, left=1.5cm, right=1.5cm, bottom=2cm}

% Font selection: Modern sans-serif font
\usepackage{helvet}
\renewcommand{\familydefault}{\sfdefault} % Use sans-serif globally

% Header and footer styling
\pagestyle{fancy}
\fancyhf{}
\fancyhead[L]{\textbf{Programmieren Tutorium}}
\fancyhead[R]{\textbf{Tobias Bück, Sam Haghighi}}
\fancyfoot[C]{\thepage}
\fancyfoot[L]{%
  \raisebox{2em}{\parbox{\textwidth}{%
      \small © Tobias Bück, CC BY 4.0.%
  }}%
}

\renewcommand{\headrulewidth}{0.4pt}
\renewcommand{\footrulewidth}{2pt}

% Title customization
\pretitle{\vspace{-2em} \begin{center}\LARGE\bfseries}
\posttitle{\end{center}\vspace{1em}}
\preauthor{}
\postauthor{}
\predate{}
\postdate{}

% Author and date styling
\renewcommand{\maketitle}{
    \begin{center}
        {\LARGE \bfseries \thetitle \par}
        \vspace{0.25em}
        {\small Datum: \today \par}
        \vspace{0.5em}
    \end{center}
}



% Custom colors and refined code style for C#
\definecolor{lightgray}{gray}{0.95}
\definecolor{darkblue}{rgb}{0, 0.2, 0.5}
\definecolor{darkgreen}{rgb}{0, 0.5, 0.2}
\definecolor{keywordcolor}{rgb}{0, 0.4, 0.8}
\definecolor{stringcolor}{rgb}{0.5, 0, 0.5}
\definecolor{commentcolor}{rgb}{0.25, 0.5, 0.25}
\definecolor{identifiercolor}{rgb}{0.6, 0.2, 0.2}

\lstset{
    language=[Sharp]C,
    basicstyle=\small\ttfamily,
    tabsize=3,
    breaklines=true,
    breakatwhitespace=true,
    frame=tb,
    framesep=6pt,
    rulecolor=\color{gray},
    backgroundcolor=\color{lightgray},
    stringstyle=\color{stringcolor}\ttfamily,
    commentstyle=\color{commentcolor}\ttfamily,
    keywordstyle=\color{keywordcolor}\bfseries,
    identifierstyle=\color{identifiercolor},
    showstringspaces=false,
    showspaces=false,
    showtabs=false,
    xleftmargin=10pt,
    framexleftmargin=10pt,
    morecomment=[l]{//},
    morecomment=[s]{/*}{*/},
    morekeywords={abstract, event, new, struct, var,
        as, explicit, null, switch, Console,
        base, extern, object, this,
        bool, false, operator, throw,
        break, finally, out, true,
        byte, fixed, override, try,
        case, float, params, typeof,
        catch, for, private, uint,
        char, foreach, protected, ulong,
        checked, goto, public, unchecked,
        class, if, readonly, unsafe,
        const, implicit, ref, ushort,
        continue, in, return, using,
        decimal, int, sbyte, virtual,
        default, interface, sealed, volatile,
        delegate, internal, short, void,
        do, is, sizeof, while,
        double, lock, stackalloc,
        else, long, static,
        enum, namespace, string},
}

\newenvironment{aufgabe}[2]
  {%
   \needspace{10\baselineskip} 
   \section*{Aufgabe #1: #2}
   \vspace{0.5em} % Increased vertical space after the section title
   \begin{adjustwidth}{1cm}{}  % Indent by 1 cm
   \setcounter{subaufgabecounter}{0} % Reset subexercise counter for each new exercise
  }
  {%
   \end{adjustwidth}
  }

\newcounter{subaufgabecounter} % Define a new counter for subexercises
\renewcommand{\thesubaufgabecounter}{\alph{subaufgabecounter})} % Format: a), b), c), ...

\newcommand{\subaufgabe}[1]{
    \needspace{6\baselineskip} 
    \vspace{0.3em} % Vertical space before each subexercise
    \par % Ensure a new paragraph (new line)
    \stepcounter{subaufgabecounter} % Increment the counter
    \textbf{\thesubaufgabecounter}~#1 % Display the label and the subexercise content
    \vspace{0.5em}
}

\newcommand{\hint}[1]{
    \vspace{0.5em}
    \par
    \textbf{\textit{Hinweis:}} #1
    \vspace{0.5em}
}

\begin{document}

\title{Programmieren Tutorium \\ \large Arbeitsblatt 8}
\author{Tobias Bück, Sam Haghighi}
\date{\today}
\maketitle

\note{Die Aufgaben sollen nicht sequentiell bearbeitet werden, suche dir die Aufgabe rauß die du bearbeiten willst!}

\begin{aufgabe}{L1}{Variablen, Schleifen, Verzweigungen}
\subaufgabe{Speichere deinen Namen, dein Alter und deine Körpergröße in Metern in Variablen und gib die Werte in der Konsole aus.}
\hint{Verwende die Datentypen \texttt{string}, \texttt{int} und \texttt{double}.}

\subaufgabe{Speichere das Alter einer Person in einer Variablen. Überprüfe mithilfe von \texttt{if}-Bedingungen, ob die Person: unter 14 Jahre alt ist, zwischen 14 und 16 Jahren alt ist, über 18 Jahre alt ist.}
\hint{Verwende \texttt{if}, \texttt{else if} und \texttt{else}, um die Bedingungen zu prüfen. Gib die passende Alterskategorie in der Konsole aus.}

\subaufgabe{Schreibe eine \texttt{for}-Schleife, die die Zahlen von 1 bis 10 in der Konsole ausgibt.}
\hint{Verwende eine \texttt{for} Schleife.}

\subaufgabe{Schreibe ein Programm, das mit einer \texttt{while}-Schleife eine Zahl so lange um 7 erhöht, bis sie mindestens den Wert 20 erreicht. Gib bei jedem Schritt die aktuelle Zahl in der Konsole aus. \\
Beispiel: beim Start mit 8
\begin{table}[h!]
\centering
\begin{tabular}{|c|c|c|}
\hline
\textbf{Wert} & \textbf{Bedingung}& \textbf{Ergebnis} \\ \hline
8             & $8 < 20$ & Ja \\ \hline
$8 + 2 = 15$)  & $15 < 20$ & Ja \\ \hline
$15 + 7 = 22$) & $22 < 20$ & Nein \\ \hline
\end{tabular}
\caption{Beispiel: Prüfung von Zahlen}
\label{tab:example}
\end{table}
}

\subaufgabe{Schreibe ein Programm, das eine Liste von Zahlen speichert. Gib mithilfe einer Schleife alle Zahlen aus der Liste aus, die gerade sind.}
\hint{Verwende den Modulo-Operator (\texttt{\%}), um zu prüfen, ob eine Zahl gerade ist.}

\end{aufgabe}

\begin{aufgabe}{S1}{structs und Algorithmen}
\hint{In C\# wird ein \texttt{struct} wie folgt definiert:}
\begin{lstlisting}
public struct NameDesStructs {
    public TYP AttributName1;
    public TYP AttributName2;
}
\end{lstlisting}
\subaufgabe{Struct \texttt{Student} mit Note}
Erstelle eine Struktur \texttt{Student} mit den Attributen \texttt{Note}, \texttt{Name}.
Die Noten sind folgendermaßen:
1.0, 1.3, ... 4.7, 5.0 wie an der HKA.


\subaufgabe{Schreibe einen Konstruktor\\
Überlege dir einen sinnvollen Konstruktor
}

\subaufgabe{Schreibe eine Methode hatBestanden
Die Methode/Funktion soll angeben ob der Student bestanden hat.
Alle Noten besser oder gleich 4 sind bestanden.
Überlege dir einen sinnvollen Rückgabewert.
Implementiere die Methode
Teste die Methode.
}
\subaufgabe{Berechne die Durchschnitts Note \\
Du hast eine Liste an Studenten \texttt{students} wie im Beispiel diehe unten.
Berechne die Durchscnittsnote (mean / arithmetisches mittel)
}
\begin{lstlisting}
List<Student> students = new List<Student>
        {
            new Student("Alice", 1.0),
            new Student("Bob", 2.3),
            new Student("Anna", 2.3),
            new Student("Charlie", 3.7),
            new Student("Jonas", 4.3),
        };
\end{lstlisting}

\subaufgabe{Berechne die beste Note \\
Du hast eine Liste an Studenten, finde heraus was die beste Note ist.
}

\subaufgabe{Berechne den Anteil an bestandenen Studenten \\
Du hast eine Liste an Studenten, berechne welcher Prozentsatz bestanden hat.
}

\subaufgabe{Berechne den Anteil an guten und sehr guten Studenten \\
gute Note => 1.7 - 2.3 (einschließlich) \\
sehr gut => 1.0 - 1.3 (einschließlich)\\
Du hast eine Liste an Studenten, berechne welcher Anteil gut oder sehr gut ist.
}


\subaufgabe{Prüfe ob bestanden \\
Du hast eine Liste an Studenten, prüfe ob der Student mit dem Namen \textit{studentName} die Klausur bestanden hat. Wenn eine Person nicht in der Liste ist, hat diese nicht an der Klausur teilgenommen und damit auch nicht bestanden.
}
\hint{Im Beispiel:
\\
Jonas -> nicht bestanden \\
Bob -> bestanden \\
Anton -> nicht teilgenommen, also nicht bestanden
}
\subaufgabe{Berechne den Mode / Modalwert (sehr schwer!!!)\\
Der Mode wird im Deutschen als Modalwert bezeichnet. Er ist ein Begriff aus der Statistik und beschreibt den Wert, der in einer Datenreihe am häufigsten vorkommt.
Du hast eine Liste an Studenten, berechne den Modalwert. Die Note die am häufigsten vorkommt.
}
\hint{Im Beispiel von oben ist der Modalwert 2.3}
\end{aufgabe}


\begin{aufgabe}{S2}{Erstelle ein \textttt{KlausurTeilnehmer} struct (schwer)}

Erstelle ein KlausurTeilnehmer-struct. Das struct soll eine Liste an Studenten als Attribut haben.

\hint{Nutze folgendes Student struct}
\begin{lstlisting}
struct Student
{
    public string Name;
    public float Grade;
}
\end{lstlisting}

\subaufgabe{Schreibe einen Konstruktor, welcher keine Argumente/Parameter übergeben bekommt}

\subaufgabe{Schreibe eine studentHinzufügen(Student person) Methode/Funktion}

\subaufgabe{Schreibe eine Methode bool hatTeilgenommen(string studentName)}

\end{aufgabe}
\end{document}
